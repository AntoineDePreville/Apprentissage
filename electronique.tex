\documentclass{article}
\usepackage{graphicx}
\usepackage{geometry}
\title{Électronique}
\geometry{margin=2.5cm}
\author{Antoine de Préville}

\begin{document}
\maketitle
\newpage
    J'utilise ici le livre d'*Électronique* rédigé par Jean-Daniel Chatelain et Roger Dessoulavy, du *Traité d'Électricité* publié sous la direction de Jacques Neirynck. J'utilise d'autres sources comme Wikipedia qui donne de bonnes informations. Si erreurs il y a, n'hésitez pas à me le faire savoir. Merci.

    \chapter{Rappel des différents composants électroniques et leurs modèles}

        \section{Préambule}
        Nous allons nous familiariser avec différents composants et de leurs modèles en électronique mais d'abord définissions quelques éléments.

            \subsection{Définitions}
                \begin{itemize}
                    \item \textbf{Semiconducteur}: Il s'agit d'un solide cristallin dont les propriétés de conduction électriques sont définies par deux bandes: la \textit{bande de valence} à qui fait correspondre des électrons impliqués dans des liaisons covalentes, et la \textit{bande de conduction} qui correspond à l'état excité des électrons et où ceux-ci peuvent se déplacer dans le cristal (d'où la conduction). Les deux bandes sont séparées par un gap qui correspond à un barrière d'énergie qui peut être franchie par une excitation extérieure. On parle de \textbf{semiconducteur intrinsèque} lorsqu'il est pur ou qu'il ne contient pas d'atome dopant.
                
                    \item \textbf{Dopage}: L'action d'ajouter des impuretés en petites quantités à une une cristal semiconducteur intrinsèque ou pur afin d'augmenter le nombre de porteurs libres. Il y a deux types de dopage:
                
                        \subitem  Le dopage P**: qui consiste à insérer un accepteur d'électrons (par rapport à la substance initiale donc un atome situé dans une colonne précédente dans le tableau périodique) dans le cristal semiconducteur. Cela va former un *trou* **p**ositivement chargé. L'insertion d'un atome accepteur d'électrons va former une pseudo-niveau d'énergie juste au-dessus de la bande de valeuce, ce qui entraîne la formation de trous.
                
                        \subitem  Le dopage N**: qui consiste à insérer un donneur d'électron (par rapport à la substance initiale donc un atome situé dans un colonne suivante dans le tableau périodique) dans le cristal semiconducteur. Cela va former une charge libre **n**égativement chargée. L'insertion d'un donneur d'électrons va faire apparaître un niveau d'énergie juste sous la bande de conduction, ce qui facilite l'excitation et par conséquent la conduction. [1]
                \end{itemize}

            \subsection{Jonction pn}
            Si chaque ensemble de porteurs (électrons et trous) se comporte comme un gaz, on peut considérer que les électrons diffusent de la région n vers la région p et inversément, de manière à utiliser tous l'espace à leur disposition. De ce fait ils créent une **zone de charge d'espace** ou zone de déplétion dans laquelle la charge volumique $\rho(x)$ n'est pas nulle, ainsi qu'un dipôle de part et d'autre de la jonction.

            La zone de charge d'espace forme un champ électrique $E(x)$ partant des charge positives vers les charges négarives et ainsi la diffusion de trous et d'électrons s'arrête car ceux-ci n'ont pas assez d'énergie pour le franchir. Ce champ électrique dérive d'un potentiel $V(x)$ qui varie dans la zone de charge d'espace. Le seuil pour franchir la zone est désigné par $U_{b0}$ à l'équilibre thermodynamique. Ainsi, en appliquant une tension $U$ entre la région p et la région n peut avoir un incident. une tension négative a pour effet d'augmenter la barrière énergétique générée par le champ électrique tandis qu'une tension $U > 0$ diminue cet écart. [2]

            \subsection{Diode à jonction}

                \subsubsection{Définitions}
                    \begin{itemize}
                        \item \textbf{Diode}: dipôle non linéaire et polarisé qui ne laisse passer le courant que dans un sens.
                        \item \textbf{Diode à jonction}: Dispositif comportant une jonction pn dont les régions p et n sont reliées à deux bornes. La tensions appliquée à ces bornes agit directement sur le champ de rétension de la diffusion et permet le contrôle du courant traversant la diode.
                        \item \textbf{Diode théorique}: est une diode qui correspond à la caractéristique exponentielle donnée par la relation 
                        \item \textbf{Courant inverse de saturation}: le courant $I$ qui circulerait en sens inverse de la diode théorique polarisée par une tension $U =-\infty$. Il dépend de la température selon la relation
                    \end{itemize}


                        \begin{equation}
                            I_s \sim T³exp(\Delta W_0/k_BT)
                        \end{equation}
        
                \subsubsection{Fonctionnement}
                    Une anode (A) est reliée à la région p et la cathode (C) est reliée à la région n le courant $I$ et la tensions $U= U_{AC}$ vont de l'anode vers la cathode. Le courant est lié à la tension par la relation 

                    \begin{equation}
                        I = I_s[exp(U/nU_T)-1]
                    \end{equation}

    \begin{thebibliography}
        @online{Wikipedia,
            author = {Wikipedia}
            title = {Dopage},
            url = {https://fr.wikipedia.org/wiki/Dopage_(semi-conducteur)},
            urldate = {2025-08-19}
        }
        
        @online{Wikipedia,
            author = {Wikipedia}
            title = {Jonction pn},
            url = {https://fr.wikipedia.org/wiki/Jonction_p-n},
            urldate = {2025-08-19}
        }
    \end{thebibliography}

\end{document}
